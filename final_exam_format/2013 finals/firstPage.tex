{\textbf \Large Final Exam} \hfill CSE 103, Fall 2014
\\

\vspace{.25in}

Name: \underline{\hspace{3in}}
\\

ID: \underline{\hspace{3.2in}}
\\

\vspace{1in}

On your desk you should have only the exam paper, writing tools, and
the cheat-sheet. The cheat-sheet is one page handwritten on both sides.

The exams are color coded. Your exam should have different color than that of your neighbours to the left, right and in front.

There are 10 questions in this exam, totalling 100 points and an
extra-credit question, worth 10 points. The final score is determined
by summing all the points and taking the min of the sum and 100.

Be clear and concise. Write your answers {\bf in the space provided}
after each question. To the degree possible, write your answers as
expressions. If you provide only the final numerical answer it would
be hard for us to assign you partial credit. Similarly, clearly
written explanations written after the final answer will make it
possible for us to give you partial credit if your final answer is
incorrect.

Do your work on the back side of the pages and then, when you are
confident of the answer, copy it, including explanations, 
neatly into the space below the question.
\vspace{0.4in}

\begin{center}
\begin{tabular}{|c|l|} \hline
\hspace{1in} & \hspace{1in} \\
1 & 10pt. \\ 
&  \\ \hline
2 & 8pt. \\
&  \\ \hline
3 & 7pt. \\
&  \\ \hline
4 & 15pt.\\
&  \\ \hline
5 & 10pt.\\
&  \\ \hline
6 & 10pt.\\
&  \\ \hline
7 & 10pt.\\
&  \\ \hline
8 & 10pt.\\
&  \\ \hline
9 & 10pt. \\
& \\ \hline
10 & 10pt. \\
& \\ \hline
EC & 10pt. \\
& \\ \hline
total & \\
& \\ \hline
\end{tabular}
\end{center}

\pagebreak
